\documentclass[twoside,twocolumn]{article}

\usepackage{blindtext} 
\usepackage{graphicx}
\usepackage[sc]{mathpazo} 
\usepackage[T1]{fontenc} 
\linespread{1.05} 
\usepackage{microtype} 


\usepackage[spanish,english]{babel} 


\usepackage[hmarginratio=1:1,top=32mm,columnsep=20pt]{geometry} 
\usepackage[hang, small,labelfont=bf,up,textfont=it,up]{caption} 
\usepackage{booktabs} 


\usepackage{lettrine} 


\usepackage{enumitem} 
\setlist[itemize]{noitemsep} 


\usepackage{abstract} 
\renewcommand{\abstractnamefont}{\normalfont\bfseries} 
\renewcommand{\abstracttextfont}{\normalfont\small\itshape} 


\usepackage{titlesec} 
\renewcommand\thesection{\Roman{section}} % 
\renewcommand\thesubsection{\roman{subsection}} 
\titleformat{\section}[block]{\large\scshape\centering}{\thesection.}{1em}{} 
\titleformat{\subsection}[block]{\large}{\thesubsection.}{1em}{} 


\usepackage{fancyhdr} 
\pagestyle{fancy} 
\fancyhead{} 
\fancyfoot{} 
\fancyhead[C]{Machine Learning Operations \today} 
\fancyfoot[RO,LE]{\thepage} 


\usepackage{titling} 

%----------------------------------------------------------------------------------------
%	TILULOS
%----------------------------------------------------------------------------------------


\setlength{\droptitle}{-4\baselineskip} 

\pretitle{\begin{center}\Huge\bfseries} 
\posttitle{\end{center}} 
\title{Machine Learning Operations} 
\author{
	Valdivia Guzman, Alejandra Maria\\
	\and
	Pazos Alarcón, Christian Joshua\\
	\and
	Farfan Colque, Mathius Omar\\
	\and
	Condori Quispe, Yhónn Joel\\
}
\date{\today} 
\renewcommand{\maketitlehookd}{
\selectlanguage{spanish} 
\begin{abstract}
\noindent 
Lorem ipsum dolor sit amet, consectetur adipiscing elit. Morbi vulputate tempus molestie. 
\end{abstract}
\selectlanguage{english} 
\begin{abstract}
\noindent 
Lorem ipsum dolor sit amet, consectetur adipiscing elit. Morbi vulputate tempus molestie. 
\end{abstract}
}

%----------------------------------------------------------------------------------------

\begin{document}

% Print the title
\maketitle

%----------------------------------------------------------------------------------------
%	Introduction
%----------------------------------------------------------------------------------------

\section{Introduction}

\lettrine[nindent=0em,lines=3]{L}orem ipsum dolor sit amet, consectetur adipiscing elit.
Morbi vulputate tempus molestie.

%----------------------------------------------------------------------------------------
%	State of Art
%----------------------------------------------------------------------------------------

\section{State of Art}

\subsection{Concept}

Machine Learning Operations, is an extension of the DevOps methodology that seeks to include the
machine learning and data science processes in the development and operations to make ML development
more reliable and productive\cite{pcmlops}. .

\subsection{Objective}

The goal of MLOps is develop, train and deploy ML models with automated procedures that integrate
data, developers, security and infrastructure teams\cite{pcmlops}. .

\subsection{Importance of MLOps}

We are in a data-driven world. The importance of the MLOps models is that they are vital to streamline
the maduration process of AI and ML projects within an organization.

The use of trained models has changed dramatically in the last decade, and now the policy of managing
ML models like ‘black boxes’ that they held in the traditional operations is no longer sufficient. Models
are updated and mantained. It’s therefore vital that the effort and time to deploy and maintain them are
reduced as much as possible\cite{pcmlops}.

\subsection{MLOps pipeline}

While there are several attempts to capture and describe MLOps, the one that is best known is the proposal
of ToughWorks\cite{multimlops}, which automates the life cycle of end-to-end Machine Learning applications. Collection,
selection and preparation of data to be used in model training, in finding and selecting the most efficient
model after testing and experimenting with different models, in developing and sending the selected model
in production\cite{symeonidis2022mlops}. 

\begin{center}
	\includegraphics[width=7cm]{./images/mlops-pipeline} 
\end{center}

\subsection{Maturity Levels}

Depending on the level of automation of a MLOps system, it can be classified at a corresponding level.
These levels were named by the community maturity levels. Although there is no universal maturity
model, the two main ones were created by Google and Microsoft\cite{symeonidis2022mlops}. 

Google model consists of three levels\cite{googlemlops}.

\begin{itemize}	
	
	\item MLOps Level 0: Manual process.
	\item MLOps Level 1: ML pipeline automation.
	\item MLOps Level 2: CI/CD pipeline automation.
	
\end{itemize}

\begin{center}
	\includegraphics[width=7cm]{./images/mlops-google} 
\end{center}

Microsoft model consists of five levels\cite{microsoftmlops}.

\begin{itemize}	
	
	\item MLOps Level 1: No MLOps.
	\item MLOps Level 2: DevOps but no MLOps.
	\item MLOps Level 3: Automated Training.
	\item MLOps Level 4: Automated Model Deployment.
	\item MLOps Level 5: Full MLOps Automated Operations.
	
\end{itemize}

\begin{center}
	\includegraphics[width=7cm]{./images/mlops-microsoft} 
\end{center}

%----------------------------------------------------------------------------------------
%	Conclusions
%----------------------------------------------------------------------------------------

\section{Conclusions}
\begin{itemize}	
	
	\item Lorem ipsum dolor sit amet, consectetur adipiscing elit. Morbi vulputate tempus molestie.
	\item Lorem ipsum dolor sit amet, consectetur adipiscing elit. Morbi vulputate tempus molestie.

\end{itemize}

%----------------------------------------------------------------------------------------
%	References
%----------------------------------------------------------------------------------------

\bibliographystyle{plain} 
\bibliography{references} 
\end{document}