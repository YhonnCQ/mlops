\documentclass[twoside,twocolumn]{article}

\usepackage{blindtext} 
\usepackage{graphicx}
\usepackage[sc]{mathpazo} 
\usepackage[T1]{fontenc} 
\linespread{1.05} 
\usepackage{microtype} 


\usepackage[spanish,english]{babel} 


\usepackage[hmarginratio=1:1,top=32mm,columnsep=20pt]{geometry} 
\usepackage[hang, small,labelfont=bf,up,textfont=it,up]{caption} 
\usepackage{booktabs} 


\usepackage{lettrine} 


\usepackage{enumitem} 
\setlist[itemize]{noitemsep} 


\usepackage{abstract} 
\renewcommand{\abstractnamefont}{\normalfont\bfseries} 
\renewcommand{\abstracttextfont}{\normalfont\small\itshape} 


\usepackage{titlesec} 
\renewcommand\thesection{\Roman{section}} % 
\renewcommand\thesubsection{\roman{subsection}} 
\titleformat{\section}[block]{\large\scshape\centering}{\thesection.}{1em}{} 
\titleformat{\subsection}[block]{\large}{\thesubsection.}{1em}{} 


\usepackage{fancyhdr} 
\pagestyle{fancy} 
\fancyhead{} 
\fancyfoot{} 
\fancyhead[C]{Machine Learning Operations \today} 
\fancyfoot[RO,LE]{\thepage} 


\usepackage{titling} 

%----------------------------------------------------------------------------------------
%	TILULOS
%----------------------------------------------------------------------------------------


\setlength{\droptitle}{-4\baselineskip} 

\pretitle{\begin{center}\Huge\bfseries} 
\posttitle{\end{center}} 
\title{Machine Learning Operations} 
\author{
	Valdivia Guzman, Alejandra Maria\\
	\and
	Pazos Alarcón, Christian Joshua\\
	\and
	Farfan Colque, Mathius Omar\\
	\and
	Condori Quispe, Yhónn Joel\\
}
\date{\today} 
\renewcommand{\maketitlehookd}{
\selectlanguage{spanish} 
\begin{abstract}
\noindent 
SIG MLOps define “una experiencia óptima de MLOps [como] aquella en la que los activos de aprendizaje
automático se tratan de forma coherente con todos los demás activos de software dentro de un entorno
de CI/CD. Los modelos de Machine Learning se pueden implementar junto con los servicios que los envuelven
y los servicios que los consumen como parte de un proceso de lanzamiento unificado”. Al codificar estas
prácticas, esperamos acelerar la adopción de ML/AI en los sistemas de software y la entrega rápida de software
inteligente. A continuación, describimos un conjunto de conceptos importantes en MLOps, como el desarrollo
incremental iterativo, la automatización, la implementación continua, el control de versiones, las pruebas, la
reproducibilidad y la supervisión . 
\end{abstract}
\selectlanguage{english} 
\begin{abstract}
\noindent 
SIG MLOps defines “an optimal MLOps experience [as] one where Machine Learning assets are treated
consistently with all other software assets within a CI/CD environment. Machine Learning models can be
deployed alongside the services that wrap them and the services that consume them as part of a unified
release process.” By codifying these practices, we hope to accelerate the adoption of ML/AI in software
systems and fast delivery of intelligent software. In the following, we describe a set of important concepts in
MLOps such as Iterative-Incremental Development, Automation, Continuous Deployment, Versioning,
Testing, Reproducibility, and Monitoring.
\end{abstract}
}

%----------------------------------------------------------------------------------------

\begin{document}

% Print the title
\maketitle

%----------------------------------------------------------------------------------------
%	Introduction
%----------------------------------------------------------------------------------------

\section{Introduction}

\lettrine[nindent=0em,lines=3]{M}achine learning models provide valuable insights to the business, but only
if those models can continuously access and analyze the organization's data. Machine learning operations
(MLOps) is the fundamental process that makes this possible.

MLOps is a cross-functional, collaborative and iterative process that puts data science capabilities to work.
To do this, MLOps treat machine learning (ML) and other types of models as reusable software artifacts.
The models can then be deployed and continuously monitored through a repeatable process.

MLOps offer continuous integration and rapid, repeatable deployment of models. As such, they help companies
discover valuable information and insights from their data faster. MLOps also include continuous monitoring
and repeatable training of models in production to ensure they perform optimally as data changes (evolves) over time.


%----------------------------------------------------------------------------------------
%	State of Art
%----------------------------------------------------------------------------------------

\section{State of Art}

\subsection{Concept}

Machine Learning Operations, is an extension of the DevOps methodology that seeks to include the
machine learning and data science processes in the development and operations to make ML development
more reliable and productive\cite{pcmlops}. .

\subsection{Objective}

The goal of MLOps is develop, train and deploy ML models with automated procedures that integrate
data, developers, security and infrastructure teams\cite{pcmlops}. .

\subsection{Importance of MLOps}

We are in a data-driven world. The importance of the MLOps models is that they are vital to streamline
the maduration process of AI and ML projects within an organization.

The use of trained models has changed dramatically in the last decade, and now the policy of managing
ML models like ‘black boxes’ that they held in the traditional operations is no longer sufficient. Models
are updated and mantained. It’s therefore vital that the effort and time to deploy and maintain them are
reduced as much as possible\cite{pcmlops}.

\subsection{MLOps pipeline}

While there are several attempts to capture and describe MLOps, the one that is best known is the proposal
of ToughWorks\cite{multimlops}, which automates the life cycle of end-to-end Machine Learning applications. Collection,
selection and preparation of data to be used in model training, in finding and selecting the most efficient
model after testing and experimenting with different models, in developing and sending the selected model
in production\cite{symeonidis2022mlops}. 

\begin{center}
	\includegraphics[width=7cm]{./images/mlops-pipeline} 
\end{center}

\subsection{Maturity Levels}

Depending on the level of automation of a MLOps system, it can be classified at a corresponding level.
These levels were named by the community maturity levels. Although there is no universal maturity
model, the two main ones were created by Google and Microsoft\cite{symeonidis2022mlops}. 

Google model consists of three levels\cite{googlemlops}.

\begin{itemize}	
	
	\item MLOps Level 0: Manual process.
	\item MLOps Level 1: ML pipeline automation.
	\item MLOps Level 2: CI/CD pipeline automation.
	
\end{itemize}

\begin{center}
	\includegraphics[width=7cm]{./images/mlops-google} 
\end{center}

Microsoft model consists of five levels\cite{microsoftmlops}.

\begin{itemize}	
	
	\item MLOps Level 1: No MLOps.
	\item MLOps Level 2: DevOps but no MLOps.
	\item MLOps Level 3: Automated Training.
	\item MLOps Level 4: Automated Model Deployment.
	\item MLOps Level 5: Full MLOps Automated Operations.
	
\end{itemize}

\begin{center}
	\includegraphics[width=7cm]{./images/mlops-microsoft} 
\end{center}

%----------------------------------------------------------------------------------------
%	Conclusions
%----------------------------------------------------------------------------------------

\section{Conclusions}
MLOps supports statistical, data science, machine learning and other types of models to deliver business value quickly.
To this end, MLOps ensure that models can be repeatedly deployed and continuously monitored.

As you have seen, the development of analytical models requires many tasks and dependencies that add complexity
and delay. The objective of MLOps is none other than to eliminate all these complexities, so that the Data Scientist can
work more efficiently, providing real value to the business in a shorter period of time. All this, through process
automation and/or simply by organizing processes in a more agile way.


%----------------------------------------------------------------------------------------
%	References
%----------------------------------------------------------------------------------------

\bibliographystyle{plain} 
\bibliography{references} 
\end{document}